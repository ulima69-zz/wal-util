\documentclass{book}

\usepackage[a6paper]{geometry}% http://ctan.org/pkg/geometry
\usepackage{fancyhdr}

\usepackage[utf8]{inputenc}
\usepackage[T1]{fontenc}
\usepackage[brazil]{babel}

 \pagestyle{fancy}
\fancyhf{}
\rhead{Latim}
\lhead{Presente do Indicativo Ativo}
%\rfoot{Page \thepage}

%\fancypagestyle{style}{
%\fancyhf{}
%\rhead{Latim}
%\lhead{Presente do Indicativo Ativo}
%\rfoot{Page \thepage}
%}

\newcommand{\radicaldesinencia}[2]{\textbf{#1}\textit{#2}}

% INICIO - Primeira Conjugação %%%%%%%%%%%%%%%%%%%%%%%%%%%%%%%%%%%%%%%%%%%%%%%%%%%%%%%%%%%%%%%%%%%%%%%%
\begin{document}
\begin{table}
\centering
\caption{Primeira Conjugação}
\vspace{0.2cm}
\begin{tabular}{c|c|c|c}
\hline
Presente		&	Infinitivo		&	Perfeito		&	Supino	\\
\hline                                    		
\radicaldesinencia{am}{o}	&	\radicaldesinencia{am}{are}	&	\radicaldesinencia{amav}{i}	&	\radicaldesinencia{amat}{um}	\\
\radicaldesinencia{d}{o}	&	\radicaldesinencia{d}{are}	&	\radicaldesinencia{ded}{i}	&	\radicaldesinencia{dat}{um}	\\
\radicaldesinencia{cogit}{o}	&	\radicaldesinencia{cogit}{are}	&	\radicaldesinencia{cogitav}{i}	&	\radicaldesinencia{cogitat}{um}	\\
 \hline
\end{tabular}
\end{table}

\begin{table}
\centering
\begin{tabular}{l|l|l|l}
\hline
1ª P. S.	&  \radicaldesinencia{am}{o}		& \radicaldesinencia{d}{o}		& \radicaldesinencia{cogit}{o} \\
2ª P. S.	&  \radicaldesinencia{am}{as} 	& \radicaldesinencia{d}{as}		& \radicaldesinencia{cogit}{as} \\
3ª P. S.	&  \radicaldesinencia{am}{at} 	& \radicaldesinencia{d}{at}		& \radicaldesinencia{cogit}{at} \\
\hline
\hline
1ª P. P.	&  \radicaldesinencia{am}{amus} 	& \radicaldesinencia{d}{amus}	& \radicaldesinencia{cogit}{amus} \\
2ª P. P. 	&  \radicaldesinencia{am}{atis} 	& \radicaldesinencia{d}{atis} 	& \radicaldesinencia{cogit}{atis} \\
3ª P. P.	&  \radicaldesinencia{am}{ant} 	& \radicaldesinencia{d}{ant}		& \radicaldesinencia{cogit}{ant} \\ 
\hline
\end{tabular}
\end{table}
% FIM - Primeira Conjugação %%%%%%%%%%%%%%%%%%%%%%%%%%%%%%%%%%%%%%%%%%%%%%%%%%%%%%%%%%%%%%%%%%%%%%%%
\clearpage





\begin{table}
\centering
\caption{Segunda Conjugação}
\vspace{0.2cm}
\begin{tabular}{c|c|c|c}
\hline
Presente		&	Infinitivo		&	Perfeito		&	Supino	\\
\hline                                    		
\radicaldesinencia{am}{o}	&	\radicaldesinencia{am}{are}	&	\radicaldesinencia{amav}{i}	&	\radicaldesinencia{amat}{um}	\\
 \hline
\end{tabular}
\end{table}

\begin{table}
\centering
\begin{tabular}{l|l}
\hline
Pessoa & Latim \\ 		% Note a separação de col. e a quebra de linhas
\hline                                    		% para uma linha horizontal
1ª P. S.	&  \textbf{am}\textit{o} \\
2ª P. S.	&  \textbf{am}\textit{as} \\
3ª P. S.	&  \textbf{am}\textit{at} \\
1ª P. P.	& \textbf{am}\textit{amus} \\
2ª P. P. 	&  \textbf{am}\textit{atis} \\
3ª P. P.	&  \textbf{am}\textit{ant} \\ \hline
\end{tabular}
\end{table}








\end{document}

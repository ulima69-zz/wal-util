\documentclass{article}

%\usepackage[brazilian]{babel}  \usepackage[T1]{fontenc}

\usepackage[utf8]{inputenc} 
\usepackage{hyperref}

\usepackage{textcomp}
% para usar em \textacutedbl




\usepackage{blindtext}

\title{Sinais diacríticos}
 \date{\the\year}
 \author{Wllyssys Alves}


\begin{document}
\maketitle

\section{Nativos}

\begin{table}[h]
\centering
\begin{tabular}{l|l|c}

\hline
\textbf{Nome} & \textbf{Comando} & \textbf{Simbolo} \\ 										% Note a separação de col. e a quebra de linhas
\hline                                    			 														% para uma linha horizontal

aspas duplas a esquerda	& \textbackslash textquotedblleft	& \textquotedblleft \\
aspas duplas a direita		& \textbackslash textquotedblright	& \textquotedblright \\
aspas simples a esquerda	& \textbackslash textquoteleft		& \textquoteleft \\
aspas simples a direita		& \textbackslash textquoteright		& \textquoteright \\
til 						& \textbackslash textasciitilde		& \textasciitilde \\
circunflexo				& \textbackslash textasciicircum		& \textasciicircum \\ \hline
grave					& \textbackslash\`{}\{o\}			& \`{o} \\
agudo					& \textbackslash\'{}\{o\}			& \'{o} \\
circumflex				& \textbackslash\^{}\{o\}			& \^{o} \\
long Hungarian umlaut		& \textbackslash H\{o\}				& \H{o} \\
til 						& \textbackslash\~{}\{o\}			& \~{o} \\
cedilha					& \textbackslash c\{o\}				& \c{c} \\
macron					& \textbackslash=\{o\}				& \={o} \\
barra abaixo 				& \textbackslash b\{o\}				& \b{o} \\
ponto acima				& \textbackslash .\{o\}				& \.{o} \\
ponto abaixo				& \textbackslash d\{o\}				& \d{o} \\
braquia					& \textbackslash u\{o\}				& \u{o} \\
v acima da letra			& \textbackslash v\{o\}				& \v{o} \\
a "tie" (inverted u) over the two letters	&	\textbackslash t\{o\}	& \t{oo} \\
umlat ou diaresis			& \textbackslash \textacutedbl\{o\} (aspas duplas no teclado)			&  \"{o} \\

\end{tabular}
\vspace{0.5cm}
\caption{Comandos com latex nativo}
\end{table}


% ftp://sunsite.icm.edu.pl/pub/CTAN/info/symbols/comprehensive/symbols-a4.pdf
% https://www.rpi.edu/dept/arc/training/latex/LaTeX_symbols.pdf



\end{document}

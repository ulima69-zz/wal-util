\documentclass[a4paper]{article}
\usepackage[utf8]{inputenc}
\usepackage[T1]{fontenc}
\usepackage[brazil]{babel}
\usepackage[margin=3cm]{geometry}
\usepackage{amsmath,amsfonts,amssymb,indentfirst}
\usepackage{booktabs,graphicx,hyperref,tikz}

\usepackage{theorem}
\theorembodyfont{\normalfont\upshape}
\newtheorem{ex}{Exemplo}

% Define comando BibTeX
\def\BibTeX{{\rm B\kern-.05em{\sc i\kern-.025em b}\kern-.08em
    T\kern-.1667em\lower.7ex\hbox{E}\kern-.125emX}}

\title{Aprendendo \LaTeX\ em 5 minutos}
\author{Regis Santos}
\date{\the\year}

\begin{document}
\maketitle

Este \'e um guia com uma r\'apida introdu\c c\~ao ao \LaTeX\ baseado em \href{http://www.howtotex.com/general/five-minute-guide-to-latex/}{Five Minute Guide to \LaTeX}, de Tim Van Der Horst e Frits Wenneker. O objetivo \'e come\c car a usar o \LaTeX\ rapidamente de uma forma bem resumida, por\'em, com os conte\'udos b\'asicos essenciais. Baixe os arquivos \href{https://bitbucket.org/rg3915/latex/downloads/exemplo5min.rar}{exemplo5min.rar} com um exemplo b\'asico.

\section{O B\'asico}

\subsection{Estrutura do documento}

Um documento \verb|tex| pode ser agrupado numa pasta principal e as figuras numa subpasta. Podemos chamar o documento de \textit{exemplo5min.tex} e as figuras de \textit{fig01.jpg, fig02.png, fig03.pdf}, etc. Temos tamb\'em o arquivo \textit{preambulo.sty} e \textit{refs.bib}.

\subsubsection{Pre\^ambulo}

O pre\^ambulo \'e a parte de configura\c c\~ao do \LaTeX. O comando \verb|\documentclass[ ]{ }| define o tipo de documento \textit{article, report, book} ou \textit{beamer}, entre outros. Dentro de \verb|[ ]| v\~ao par\^ametros opcionais como, \textit{10pt, 11pt, 12pt, a4paper, twoside}, etc.

\begin{ex}
Relat\'orio com fonte 10pt em papel A4

\begin{verbatim}
\documentclass[a4paper,10pt]{report}
\end{verbatim} 

Al\'em disso, alguns pacotes s\~ao fundamentais.

\begin{verbatim}
\usepackage[utf8]{inputenc}
\usepackage[T1]{fontenc}
\usepackage[brazil]{babel}
\end{verbatim} 

E o t\'itulo e autor.

\begin{verbatim}
\title{Aprendendo \LaTeX\ em 5 minutos}
\author{Seu nome}
\date{\today}
\end{verbatim} 

\end{ex}

\subsubsection{Ambientes}

Todo ambiente come\c ca com \verb|\begin{}| e termina com \verb|\end{}|. O ambiente \textit{documento} \'e o que retorna a parte que ser\'a impressa.

\begin{verbatim}
\begin{document}
  ... texto do documento aqui
\end{document}
\end{verbatim}

Para mostrar o t\'itulo do documento digite \verb|\maketitle| logo depois de \verb|\begin{document}|.

\subsubsection{Se\c c\~oes}

Um documento pode ser dividido em partes como: \verb|\part{}|, \verb|\chapter{}|, \verb|\section{}|, \verb|\subsection{}|, \verb|\subsubsection{}|. Coloque o t\'itulo entre \verb|{}|. A classe \textit{article} n\~ao aceita cap\'itulo, somente \textit{report} ou \textit{book} aceitam cap\'itulos.

\begin{ex}
Exemplo de cap\'itulo
\footnote{OBS: A palavra \textit{seção} foi escrita na forma padrão do \LaTeX\ para evitar conflitos de caracteres para o padrão ISO e UTF8. Mais informações em \href{http://latexbr.blogspot.com.br/2011/02/acentos-e-caracteres-especiais.html}{Acentos e caracteres especiais}.}

\begin{verbatim}
\chapter{Primeiro Capitulo}

\section{Primeira Se\c c\~ao}
\end{verbatim} 

\end{ex}

\subsubsection{Sum\'ario}

No caso de documentos com cap\'itulos, para mostrar o sum\'ario digite \verb|\tableofcontents| dentro de \verb|\begin{document}|.
                                                   
\subsection{Formatando o texto}

Os comandos b\'asicos s\~ao: \verb|\textit{italico}|, \verb|\textbf{negrito}| e \verb|\underline{sublinhado}|. Para que apare\c ca caracteres especiais no documento digite, por exemplo:\\
\verb|\$|, \verb|\%|, \verb|\&|. Uma linha \'e quebrada com \verb|\\|, e uma p\'agina com \verb|\newpage|.

\subsection{Refer\^encias}

Uma refer\^encia \'e usada a partir de uma \verb|\label{}| com o comando \verb|\ref{}|.

\begin{ex}
Exemplo de refer\^encia

\begin{verbatim}
\subsection{Outra Se\c c\~ao}
\label{sec:nome_da_secao}

Veja o exemplo na se\c c\~ao \ref{sec:nome_da_secao}
\end{verbatim}
\end{ex}

\section{Conte\'udo}

\subsection{Equa\c c\~oes}

Para escrever uma equa\c c\~ao matem\'atica numa linha digite entre \verb|$ $|.

\begin{ex}
Equa\c c\~ao na linha de texto.
\begin{verbatim}
Dado $f(x) = ax^2 + bx + c$, as raizes s\~ao obtidas fazendo $f(x) = 0$.
\end{verbatim} 

Dado $f(x) = ax^2 + bx + c$, as raizes s\~ao obtidas fazendo $f(x) = 0$.
\end{ex}

Para escrever uma equa\c c\~ao destacada digite entre \verb|\[ \]|.

\begin{ex}
Equa\c c\~ao destacada do texto.

\begin{verbatim}
\[
x = \frac{-b \pm \sqrt{b^2 - 4ac}}{2a}
\]
\end{verbatim} 

\[
x = \frac{-b \pm \sqrt{b^2 - 4ac}}{2a}
\]

\end{ex}

\begin{ex}
A equa\c c\~ao a seguir \'e composta de m\'ultiplas linhas. Requer o pacote \verb|\usepackage{amsmath}|.

\begin{verbatim}
\[
\begin{gathered}
  5x - 15 = 0 \hfill \\
  5x = 15 \hfill \\
  x = 3 \hfill \\
\end{gathered}
\]
\end{verbatim} 

\[
\begin{gathered}
  5x - 15 = 0 \hfill \\
  5x = 15 \hfill \\
  x = 3 \hfill \\
\end{gathered}
\]
\end{ex}

\begin{ex}
Para representar os n\'umeros inteiros, reais, complexos carregue o pacote \verb|\usepackage{amsfonts}|.

\begin{tabular}{lll}
  N\'umeros inteiros:  & $\mathbb{Z}$ & \verb|$\mathbb{Z}$| \\
  N\'umeros reais:     & $\mathbb{R}$ & \verb|$\mathbb{R}$| \\
  N\'umeros complexos: & $\mathbb{C}$ & \verb|$\mathbb{C}$| \\
\end{tabular}

\end{ex}

\begin{ex}
Alguns s\'imbolos matem\'aticos requer o pacote \verb|\usepackage{amssymb}|.

\verb|$\nexists, \leqslant, \varnothing$|

$\nexists \qquad \leqslant \qquad \varnothing$
\end{ex}

\begin{ex}
Equa\c c\~ao com numera\c c\~ao

\begin{verbatim}
\begin{equation}
\int_a^b f(x)dx = \mathop {\lim }\limits_{n \to \infty }
  \sum_{i = 1}^n f(x_i^*) \Delta x
\end{equation}
\end{verbatim} 

\begin{equation}
\int_a^b f(x)dx = \mathop {\lim }\limits_{n \to \infty }  
\sum_{i = 1}^n f(x_i^*) \Delta x
\end{equation}

\end{ex}

\begin{ex}
Equa\c c\~ao sem numera\c c\~ao

\begin{verbatim}
\begin{equation*}
\int_r F(r) dr = \int_a^b F(r(t)).r'(t) dt
\end{equation*}
\end{verbatim} 

\begin{equation*}
\int_r F(r) dr = \int_a^b F(r(t)).r'(t) dt
\end{equation*}

\end{ex}

Mais informa\c c\~oes em \href{http://mirror.ctan.org/info/math/voss/mathmode/Mathmode.pdf}{mathmode}.

\subsection{Nota de rodap\'e}

Para inserir uma nota de rodap\'e digite \verb|\footnote{Nota de rodap\'e}|\footnote{Nota de rodap\'e}.

\newpage

\subsection{Figuras}

Carregue o pacote \verb|graphicx| e use a seguinte sintaxe:

\begin{verbatim}
\begin{figure}[h]
\centering
\includegraphics[width=5cm]{figuras/abelha}
\caption{Exemplo de figura JPG.}
\label{fig:abelha}
\end{figure}
\end{verbatim}

\begin{figure}[h]
  \centering
  \includegraphics[width=5cm]{figuras/abelha}
  \caption{Exemplo de figura JPG.}
  \label{fig:abelha}
\end{figure}

\begin{verbatim}
\begin{figure}[h]
  \centering
  \begin{tikzpicture}
    \draw[fill=yellow] (0,0) circle (1);
    \draw[blue,->] (0,0) -- (45:1);
    \begin{scope}[scale=.5,shift={(10,0)}]
      \draw[blue,domain=-2*pi:2*pi,samples=200] plot (\x,{sin(\x r)});
    \end{scope}
  \end{tikzpicture}
  \caption{Exemplo de figura Tikz.}
  \label{fig:tikz}
\end{figure} 
\end{verbatim} 

\begin{figure}[h]
  \centering
  \begin{tikzpicture}
    \draw[fill=yellow] (0,0) circle (1);
    \draw[blue,->] (0,0) -- (45:1);
    \begin{scope}[scale=.5,shift={(10,0)}]
      \draw[blue,domain=-2*pi:2*pi,samples=200] plot (\x,{sin(\x r)});
    \end{scope}
  \end{tikzpicture}   
  \caption{Exemplo de figura Tikz.}
  \label{fig:tikz}
\end{figure}

O comando \verb|\includegraphics{}| aceita figuras JPG, PNG e PDF. O \href{http://latexbr.blogspot.com.br/search/label/TikZ}{TikZ} \'e um pacote que desenha figuras vetoriais e gr\'aficos de fun\c c\~oes.

\newpage

\subsection{Tabelas}

Para tabelas \'e recomend\'avel usar o pacote \verb|booktabs|.

\begin{table}[h]
  \centering
  \begin{tabular}{crl}
    \toprule
    Nome & Nota & Ano \\
    \midrule
    Felipe & 7.5 & 2012\\
    Ricardo & 2 & 2010\\
    \bottomrule
  \end{tabular}
  \caption{Exemplo de tabela}
  \label{tab:tabela}
\end{table}

\begin{verbatim}
\begin{table}[h]
  \centering
  \begin{tabular}{crl}
    \toprule
    Nome & Nota & Ano \\
    \midrule
    Felipe & 7.5 & 2012\\
    Ricardo & 2 & 2010\\
    \bottomrule
  \end{tabular}
  \caption{Exemplo de tabela}
  \label{tab:tabela}
\end{table}
\end{verbatim} 

\subsection{Listas}

\begin{ex}
Para criar uma lista numerada carregue o pacote \verb|enumerate|.

\begin{verbatim}
\begin{enumerate}
 \item Primeiro
 \item Segundo
 \item Terceiro
 \end{enumerate} 
\end{verbatim} 

\begin{enumerate}
 \item Primeiro
 \item Segundo
 \item Terceiro
\end{enumerate}

\end{ex}

Experimente \verb|\begin{enumerate}[a)]|.

\begin{ex}
Para listas sem numera\c c\~ao use:

\begin{verbatim}
\begin{itemize}
 \item Primeiro
 \item Segundo
 \item Terceiro
\end{itemize}
\end{verbatim} 

\begin{itemize}
 \item Primeiro
 \item Segundo
 \item Terceiro
\end{itemize}

\end{ex}

\newpage

\section{Bibliografia com \BibTeX}

\subsection{Usando \BibTeX}

Para usar a \BibTeX\ digite no final do documento, antes de \verb|\end{document}|.

\begin{verbatim}
\bibliographystyle{abbrv}
\bibliography{refs}
\end{verbatim}

\subsection{Adicionando itens a bibliografia}

Pesquise as refer\^encias em \href{http://scholar.google.com.br/schhp?hl=pt-BR}{Google Acad\^emicos}. Clique em Configura\c c\~oes (lado superior direito da tela) e marque 'Mostre links para importar cita\c c\~oes para o \BibTeX'. A partir dai pesquise a refer\^encia e clique em 'Importe para o \BibTeX'.

\'E necess\'ario que a refer\^encia seja citada no documento, para isso digite \verb|\cite{}|. \cite{oetiker}, \cite{lamport}, \cite{mittelbach}, \cite{tantau}
Veja em \textit{refs.bib} algumas refer\^encias.

Para compilar com refer\^encia abra o terminal e digite:

\begin{verbatim}
pdflatex exemplo
bibtex exemplo
pdflatex exemplo
pdflatex exemplo
\end{verbatim} 

Veja também \href{http://latexbr.blogspot.com.br/2012/07/cartao-com-principais-comandos-do-latex.html}{latexrefcard}.

\href{http://latexbr.blogspot.com.br/}{LaTeXBR}

\bibliographystyle{abbrv}
\bibliography{refs}

\end{document}
\documentclass{article}

%\usepackage[brazilian]{babel}  \usepackage[T1]{fontenc}

\usepackage[utf8]{inputenc} 
\usepackage{hyperref}


\usepackage{blindtext}

%\title{Apologia de Sócrates} \date{2013-09-01} \author{Platão}

\DeclareUnicodeCharacter{1F701}{asdad}
\begin{document}

%%% runas 

%%%% http://www.personal.psu.edu/ejp10/blogs/gotunicode/charts/runes.html

%% https://www.wikiwand.com/en/Franks_Casket

%https://www.rpi.edu/dept/arc/training/latex/LaTeX_symbols.pdf


Teste 




Da wikipedia

In this chapter we will tackle matters related to input encoding, typesetting diacritics and special characters.

In the following document, we will refer to special characters for all symbols other than the lowercase letters a–z, uppercase letters A-Z, figures 0–9, and English punctuation marks.

Some languages usually need a dedicated input system to ease document writing. This is the case for Arabic, Chinese, Japanese, Korean and others. This specific matter will be tackled in Internationalization.

The rules for producing characters with diacritical marks, such as accents, differ somewhat depending whether you are in text mode, math mode, or the tabbing environment.



Input encoding

TeX uses ASCII by default. But 128 characters is not enough to support non-english languages. TeX has its own way to do that with commands for every diacritical marking (see Escaped codes). But if we want accents and other special characters to appear directly in the source file, we have to tell TeX that we want to use a different encoding.

There are several encodings available to LaTeX:

    ASCII: the default. Only bare english characters are supported in the source file.
    ISO-8859-1 (a.k.a. Latin 1): 8-bits encoding. It supports most characters for latin languages, but that's it.
    UTF-8: a Unicode multi-byte encoding. Supports the complete Unicode specification.
    Others...

In the following we will assume you want to use UTF-8.

There are some important steps to specify encoding.

    Make sure your text editor decodes the file in UTF-8.
    Make sure it saves your file in UTF-8. Most text editors do not make the distinction, but some do, such as Notepad++.
    If you are working in a terminal, make sure it is set to support UTF-8 input and output. Some old Unix terminals may not support UTF-8. PuTTY is not set to use UTF-8 by default, you have to configure it.
    Tell LaTeX that the source file is UTF-8 encoded.
    
    
    Unicode keyboard input
Wikipedia-logo.png 	

Wikipedia has related information at Unicode input

Some operating systems provide a keyboard combination to input any Unicode code point, the so-called unicode compose key.

Many X applications (*BSD and GNU/Linux) support the Ctrl+Shift+u combination. A 'u' symbol should appear. Type the code point and press enter or space to actually print the character. Example:

<Ctrl+Shift+u> 20AC <space>

will print the euro character.

Desktop environments like GNOME and KDE may feature a customizable compose key for more memorizable sequences.

Xorg features advanced keyboard layouts with variants that let you enter a lot of characters easily with combination using the aprioriate modifier, like Alt Gr. It highly depends on the selected layout+variant, so we suggest you to play a bit with your keyboard, preceeding every key and dead key with the Alt Gr modifier.

In Windows, you can hold Alt and type a <codepoint> to get a desired character. For example,

<Alt> + 0252

will print the German letter ü.

\begin{verbatim}
\usepackage[utf8]{inputenc}
\end{verbatim}


\begin{enumerate}  
	\item grave: \textbackslash\`{}\{o\}  produces \`{o} 
	\item \textbackslash\'{}\{o\} produces an acute accent,     \'{o} 
	\item \textbackslash\^{}\{o\} produces a circumflex,     \^{o} 

	%corrigir
	\item \textbackslash \''{} \{o\} produces an umlaut or dieresis,   \"{o} 
	
	
	\item \textbackslash H \{o\} produces a long Hungarian umlaut    \H{o} 
		
		
	\item \textbackslash\~{}\{o\} produces a tilde,     \~{o} 
			
			
	\item \textbackslash c\{o\} produces a cedilla,     \c{c} 
				
				
	\item \textbackslash=\{o\}  produces a macron accent (a bar over the letter)    \={o}
					
					
	\item \textbackslash b\{o\} produces a bar under the letter    \b{o} 
						
						
	\item \textbackslash .\{o\} produces a dot over the letter    \.{o}
	
	
	\item \textbackslash d\{o\} produces a dot under the letter     \d{o} 
	
	\item \textbackslash u\{o\} produces a breve over the letter    \u{o} 
	
	\item \textbackslash v\{o\} produces a "v" over the letter    \v{o} 
	
	\item \textbackslash t\{o\} produces a "tie" (inverted u) over the two letters    \t{oo} 
	

\end{enumerate}

\url{https://en.wikibooks.org/wiki/LaTeX/Special_Characters}\\
\href{https://en.wikibooks.org/wiki/LaTeX/Special_Characters}{Wikibooks home}

\begin{description}
\item [Ant] \blindtext
\item [Elephant] \blindtext
\end{description}

Ver \url{http://www.unicode.org/charts/#symbols}

Alquimia \url{http://www.unicode.org/charts/PDF/U1F600.pdf}

%%%%%

\begin{verbatim}

Accents

The rules differ somewhat depending whether you are in text mode, math mode, or the tabbing environment
Text mode

The following accents may be placed on letters. Although "o" is used in most of the example, the accents may be placed on any letter. Accents may even be placed above a "missing" letter; for example, \~{} produces a tilde over a blank space.

The following commands may be used only in paragraph or LR mode.

    \`{o} produces a grave accent, ò
    \'{o} produces an acute accent, ó
    \^{o} produces a circumflex, ô
    \"{o} produces an umlaut or dieresis, ö
    \H{o} produces a long Hungarian umlaut
    \~{o} produces a tilde, õ
    \c{c} produces a cedilla, ç
    \={o} produces a macron accent (a bar over the letter)
    \b{o} produces a bar under the letter
    \.{o} produces a dot over the letter
    \d{o} produces a dot under the letter
    \u{o} produces a breve over the letter
    \v{o} produces a "v" over the letter
    \t{oo} produces a "tie" (inverted u) over the two letters

Note that the letters "i" and "j" require special treatment when they are given accents because it is often desirable to replace the dot with the accent. For this purpose, the commands \i and \j can be used to produce dotless letters.

For example,

    \^{\i} should be used for i, circumflex, î
    \"{\i} should be used for i, umlaut, ï

Math mode

Several of the above and some similar accents can also be produced in math mode. The following commands may be used only in math mode.

    \hat{o} is similar to the circumflex (cf. \^)
    \widehat{oo} is a wide version of \hat over several letters
    \check{o} is a vee or check (cf. \v)
    \tilde{o} is a tilde (cf. \~)
    \widetilde{oo} is a wide version of \tilde over several letters
    \acute{o} is an acute accent (cf. \`)
    \grave{o} is a grave accent (cf. \')
    \dot{o} is a dot over the letter (cf. \.)
    \ddot{o} is a double dot over the letter
    \breve{o} is a breve (cf. \u)
    \bar{o} is a macron (cf. \=)
    \vec{o} is a vector (arrow) over the letter

Tabbing environment

Some of the accent marks used in running text have other uses in the tabbing environment. In that case they can be created with the following command:

    \a' for an acute accent
    \a` for a grave accent
    \a= for a macron accent
    
    
    
    %%%%%%%%%%%%%%%%%%%%%
    
    Greek letters

These commands may be used only in math mode.
Lower case

    \alpha
    \beta
    \gamma
    \delta
    \epsilon
    \varepsilon (variation, script-like)
    \zeta
    \eta
    \theta
    \vartheta (variation, script-like)
    \iota
    \kappa
    \lambda
    \mu
    \nu
    \xi
    \pi
    \varpi (variation)
    \rho
    \varrho (variation, with the tail)
    \sigma
    \varsigma (variation, script-like)
    \tau
    \upsilon
    \phi
    \varphi (variation, script-like)
    \chi
    \psi
    \omega

Capital letters

    \Gamma
    \Delta
    \Theta
    \Lambda
    \Xi
    \Pi
    \Sigma
    \Upsilon
    \Phi
    \Psi
    \Omega

%%%%%%%%%

Non-English characters and ligatures

These commands may be used only in paragraph or LR mode.

    \ae - small ae ligature (diphthong), æ
    \AE - capital ae ligature, Æ
    \oe - small oe ligature
    \OE - capital OE ligature
    \aa - small a, ring, å
    \AA - capital A, ring, Å
    \o - small o, slash, ø
    \O - capital O, slash, Ø
    \ss - German sz ligature, ß

Non-English punctuation

An inverted "question mark" is made by ?`. It will also be generated by a > in text mode.

An inverted "exclamation point" is made by !`. It will also be generated by a < in text mode.


%%%%%%



Miscellaneous symbols
Some symbols for math

The following symbols are also used only in math mode

    \aleph - Hebrew aleph
    \hbar - h-bar, Planck's constant
    \imath - variation on "i"; no dot
        See also \i
    \jmath - variation on "j"; no dot
        See also \j
    \ell - script (loop) "l"
    \wp - fancy script lowercase "P"
    \Re - script capital "R" ("Real")
    \Im - script capital "I" ("Imaginary")
    \prime - prime (also obtained by typing ')
    \nabla - inverted capital Delta
    \surd - radical (square root) symbol
    \angle - angle symbol
    \forall - "for all" (inverted A)
    \exists - "exists" (left-facing E)
    \backslash - backslash
    \partial - partial derivative symbol
    \infty - infinity symbol
    \triangle - open triangle symbol
    \Box - open square
    \Diamond - open diamond

    \flat - music: flat symbol
    \natural - music: natural symbol
    \sharp - music: sharp symbol

    \clubsuit - playing cards: club suit symbol
    \diamondsuit - playing cards: diamond suit symbol
    \heartsuit - playing cards: heart suit symbol
    \spadesuit - playing cards: spade suit symbol

Some other symbols

The following symbols can be used in any mode.

    \dag - dagger
    \ddag - double dagger
    \S - section symbol
    \P - paragraph symbol
    \copyright - copyright symbol
    \pounds - British pound sterling symbol

Calligraphic style letters

Twenty-six calligraphic letters are provided (the upper case alphabet). These can only be used in math mode.

In LaTeX 2.09 they are produced with the \cal declaration

    ${\cal A}$

In LaTeX2e they are obtained with the \mathcal command:

    $\mathcal{CAL}$
%%%%%%%%%%%%


Ellipses

Ellipses (three dots) can be produced by the following commands

    \ldots - horizontally at bottom of line
    \cdots - horizontally center of line (math mode only)
    \ddots - diagonal (math mode only)
    \vdots - vertical (math mode only)


\end{verbatim}
\href{https://www.giss.nasa.gov/tools/latex/ltx-401.html}{Nasa}



\end{document}

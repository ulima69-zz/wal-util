\documentclass[10pt,a4paper,landscape]{article}
\usepackage{multicol}
\usepackage{calc}
\usepackage{ifthen}
\usepackage[landscape]{geometry}

% Para gerar este documento execute
%  pdflatex latexrefcard.tex

% 2012-06
% Traduzido e adaptado para o Portugues por Regis Santos \copyright\ 2012
% a partir do original de Winston Chang \copyright\ 2012.

% http://www.stdout.org/$\sim$winston/latex/

% Para fazer:
% \listoffigures \listoftables
% \setcounter{secnumdepth}{0}


% As margens est\~ao configuradas para .5 polegada se usar letter paper,
% e para 1 cm se usar papel A4.
% Se usar outro papel diferente, use 1 cm de margem por padr\~ao.
\ifthenelse{\lengthtest { \paperwidth = 11in}}
	{ \geometry{top=.5in,left=.5in,right=.5in,bottom=.5in} }
	{\ifthenelse{ \lengthtest{ \paperwidth = 297mm}}
		{\geometry{top=1cm,left=1cm,right=1cm,bottom=1cm} }
		{\geometry{top=1cm,left=1cm,right=1cm,bottom=1cm} }
	}

% Desativa cabecalho e rodape
\pagestyle{empty}
 

% Redefine comandos de se\c c\~ao para usar menos espa\c co
\makeatletter
\renewcommand{\section}{\@startsection{section}{1}{0mm}%
                                {-1ex plus -.5ex minus -.2ex}%
                                {0.5ex plus .2ex}%x
                                {\normalfont\large\bfseries}}
\renewcommand{\subsection}{\@startsection{subsection}{2}{0mm}%
                                {-1explus -.5ex minus -.2ex}%
                                {0.5ex plus .2ex}%
                                {\normalfont\normalsize\bfseries}}
\renewcommand{\subsubsection}{\@startsection{subsubsection}{3}{0mm}%
                                {-1ex plus -.5ex minus -.2ex}%
                                {1ex plus .2ex}%
                                {\normalfont\small\bfseries}}
\makeatother

% Define comando BibTeX
\def\BibTeX{{\rm B\kern-.05em{\sc i\kern-.025em b}\kern-.08em
    T\kern-.1667em\lower.7ex\hbox{E}\kern-.125emX}}

% N\~ao imprime numero de se\c c\~ao
\setcounter{secnumdepth}{0}


\setlength{\parindent}{0pt}
\setlength{\parskip}{0pt plus 0.5ex}


% -----------------------------------------------------------------------

\begin{document}

\raggedright
\footnotesize
\begin{multicols}{3}


% Parametros para multicolunas
\setlength{\premulticols}{1pt}
\setlength{\postmulticols}{1pt}
\setlength{\multicolsep}{1pt}
\setlength{\columnsep}{2pt}

\begin{center}
     \Large{\textbf{\LaTeXe\ Principais Comandos}} \\
\end{center}

\section{Classes de documentos}
\begin{tabular}{@{}ll@{}}
\verb!book!    & Padr\~ao s\~ao dois lados. \\
\verb!report!  & Sem divis\~ao por \verb!\part!. \\
\verb!article! & Sem divis\~ao \verb!\part! ou \verb!\chapter!. \\
\verb!letter!  &  \\
\verb!slides!  & Fonte larga sans-serif.
\end{tabular}

Usado para iniciar um documento:
\verb!\documentclass{!\textit{classe}\verb!}!.  Use
\verb!\begin{document}! para iniciar e \verb!\end{document}! para 
finalizar o documento.

\subsection{Op\c c\~oes comuns para \texttt{documentclass}}
\newlength{\MyLen}
\settowidth{\MyLen}{\texttt{letterpaper}/\texttt{a4paper} \ }
\begin{tabular}{@{}p{\the\MyLen}%
                @{}p{\linewidth-\the\MyLen}@{}}
\texttt{10pt}/\texttt{11pt}/\texttt{12pt} & Tamanho da fonte. \\
\texttt{letterpaper}/\texttt{a4paper} & Tamanho do papel. \\
\texttt{twocolumn} & Usa duas colunas. \\
\texttt{twoside}   & Define margens para frente e verso. \\
\texttt{landscape} & Orienta\c c\~ao paisagem. Pode usar 
                     \texttt{dvips -t landscape}. \\
\texttt{draft}     & Linhas com espa\c camento duplo.
\end{tabular}

Uso:
\verb!\documentclass[!\textit{opt,opt}\verb!]{!\textit{classe}\verb!}!.


\subsection{Pacotes}
\settowidth{\MyLen}{\texttt{multicol} }
\begin{tabular}{@{}p{\the\MyLen}%
                @{}p{\linewidth-\the\MyLen}@{}}
%\begin{tabular}{@{}ll@{}}
\texttt{fullpage}  &  Usa 1 polegada de margem. \\
\texttt{anysize}   &  Define margens: \verb!\marginsize{!\textit{l}%
                        \verb!}{!\textit{r}\verb!}{!\textit{t}%
                        \verb!}{!\textit{b}\verb!}!.            \\
\texttt{multicol}  &  Usa $n$ colunas: 
                        \verb!\begin{multicols}{!$n$\verb!}!.   \\
\texttt{latexsym}  &  Usa s\'imbolos \LaTeX. \\
\texttt{graphicx}  &  Exibe imagem:
                        \verb!\includegraphics[width=!%
                        \textit{x}\verb!]{!%
                        \textit{arquivo}\verb!}!. \\
\texttt{url}       & Insere URL: \verb!\url{!%
                        \textit{http://\ldots}%
                        \verb!}!.
\end{tabular}

Use antes de \verb!\begin{document}!. 
Uso: \verb!\usepackage{!\textit{pacote}\verb!}!


\subsection{T\'itulo}
\settowidth{\MyLen}{\texttt{.author.text.} }
\begin{tabular}{@{}p{\the\MyLen}%
                @{}p{\linewidth-\the\MyLen}@{}}
\verb!\author{!\textit{texto}\verb!}! & Autor do documento. \\
\verb!\title{!\textit{texto}\verb!}!  & T\'itulo do documento. \\
\verb!\date{!\textit{texto}\verb!}!   & Data. Ex: \verb!\date{\today}!,\verb!\date{}! \\
\end{tabular}

Esses comandos vem antes de \verb!\begin{document}!. A declara\c c\~ao
\verb!\maketitle! retorna o t\'itulo no topo do documento.

\subsection{Miscel\^anea}
\settowidth{\MyLen}{\texttt{.pagestyle.empty.} }
\begin{tabular}{@{}p{\the\MyLen}%
                @{}p{\linewidth-\the\MyLen}@{}}
\verb!\pagestyle{empty}!     &   Cabe\c calho e rodap\'e vazio
                                 e p\'agina sem numera\c c\~ao. \\
\verb!\tableofcontents!      &   Adiciona o sum\'ario. \\

\end{tabular}



\section{Estrutura do documento}
\begin{multicols}{2}
\verb!\part{!\textit{t\'itulo}\verb!}!  \\
\verb!\chapter{!\textit{t\'itulo}\verb!}!  \\
\verb!\section{!\textit{t\'itulo}\verb!}!  \\
\verb!\subsection{!\textit{t\'itulo}\verb!}!  \\
\verb!\subsubsection{!\textit{t\'itulo}\verb!}!  \\
\verb!\paragraph{!\textit{t\'itulo}\verb!}!  \\
\verb!\subparagraph{!\textit{t\'itulo}\verb!}!
\end{multicols}
{\raggedright
Usando \verb!\setcounter{secnumdepth}{!$x$\verb!}! suprime n\'umeros dos subn\'iveis $>x$, onde \verb!chapter! \'e n\'ivel 0.
Use \texttt{*}, numa \verb!\section*{!\textit{t\'itulo}\verb!}!,
para n\~ao numerar um item particular---este item n\~ao ir\'a aparecer no sum\'ario.
}

\subsection{Ambientes de texto}
\settowidth{\MyLen}{\texttt{.begin.quotation.}}
\begin{tabular}{@{}p{\the\MyLen}%
                @{}p{\linewidth-\the\MyLen}@{}}
\verb!\begin{comment}!    &  Coment\'ario (n\~ao imprim\'ivel). Requer o pacote \texttt{verbatim}. \\
\verb!\begin{quote}!      &  Indenta um bloco de cita\c c\~ao. \\
\verb!\begin{quotation}!  &  \texttt{quote} com par\'agrafo recuado. \\
\verb!\begin{verse}!      &  Bloco de cita\c c\~ao para versos.
\end{tabular}

\subsection{Listas}
\settowidth{\MyLen}{\texttt{.begin.description.}}
\begin{tabular}{@{}p{\the\MyLen}%
                @{}p{\linewidth-\the\MyLen}@{}}
\verb!\begin{enumerate}!        &  Lista numerada. \\
\verb!\begin{itemize}!          &  Lista com marca\c c\~ao. \\
\verb!\begin{description}!      &  Lista com descri\c c\~ao. \\
\verb!\item! \textit{texto}      &  Adiciona um item. \\
\verb!\item[!\textit{x}\verb!]! \textit{texto}
                                &  Use \textit{x} em vez de marca\c c\~ao normal ou n\'umero. Necess\'ario para descri\c c\~ao.\\
\end{tabular}




\subsection{Refer\^encias}
\settowidth{\MyLen}{\texttt{.pageref.marker..}}
\begin{tabular}{@{}p{\the\MyLen}%
                @{}p{\linewidth-\the\MyLen}@{}}
\verb!\label{!\textit{marcador}\verb!}!   &  Define uma marca para refer\^encia cruzada, 
                          geralmente \'e da forma \verb!\label{sec:item}!. \\
\verb!\ref{!\textit{marcador}\verb!}! &  Retorna n\'umero da se\c c\~ao do marcador. \\
\verb!\pageref{!\textit{marcador}\verb!}! &  Retorna n\'umero da p\'agina do marcador. \\
\verb!\footnote{!\textit{texto}\verb!}!  &  Imprime nota de rodap\'e na parte inferior da p\'agina. \\
\end{tabular}


\subsection{Objetos flutuantes}
\settowidth{\MyLen}{\texttt{.begin.equation..place.}}
\begin{tabular}{@{}p{\the\MyLen}%
                @{}p{\linewidth-\the\MyLen}@{}}
\verb!\begin{table}[!\textit{lugar}\verb!]!     &  Adiciona tabela numerada. \\
\verb!\begin{figure}[!\textit{lugar}\verb!]!    &  Adiciona figura numerada. \\
\verb!\begin{equation}[!\textit{lugar}\verb!]!  &  Adiciona equa\c c\~ao numerada. \\
\verb!\caption{!\textit{texto}\verb!}!           & Legenda para o objeto. \\
\end{tabular}

O \textit{lugar} \'e uma lista de posi\c c\~oes v\'alidas para o objeto.  \texttt{t}=topo,
\texttt{h}=aqui, \texttt{b}=embaixo, \texttt{p}=p\'agina separada, \texttt{!}=neste lugar mesmo que fique feio. Legendas e etiquetas de marcadores devem estar dentro do ambiente.

%---------------------------------------------------------------------------

\section{Propriedades do texto}

\subsection{Fonte}
\newcommand{\FontCmd}[3]{\PBS\verb!\#1{!\textit{text}\verb!}!  \> %
                         \verb!{\#2 !\textit{text}\verb!}! \> %
                         \#1{#3}}
\begin{tabular}{@{}l@{}l@{}l@{}}
\textit{Comando} & \textit{Declara\c c\~ao} & \textit{Efeito} \\
\verb!\textrm{!\textit{texto}\verb!}!                    & %
        \verb!{\rmfamily !\textit{texto}\verb!}!               & %
        \textrm{Fam\'ilia Romana} \\
\verb!\textsf{!\textit{texto}\verb!}!                    & %
        \verb!{\sffamily !\textit{texto}\verb!}!               & %
        \textsf{Fam\'ilia Sem serifa} \\
\verb!\texttt{!\textit{texto}\verb!}!                    & %
        \verb!{\ttfamily !\textit{texto}\verb!}!               & %
        \texttt{Fam\'ilia M\'aquina de escrever} \\
\verb!\textmd{!\textit{texto}\verb!}!                    & %
        \verb!{\mdseries !\textit{texto}\verb!}!               & %
        \textmd{S\'erie m\'edia} \\
\verb!\textbf{!\textit{texto}\verb!}!                    & %
        \verb!{\bfseries !\textit{texto}\verb!}!               & %
        \textbf{S\'erie negrito} \\
\verb!\textup{!\textit{texto}\verb!}!                    & %
        \verb!{\upshape !\textit{texto}\verb!}!               & %
        \textup{Forma em p\'e} \\
\verb!\textit{!\textit{texto}\verb!}!                    & %
        \verb!{\itshape !\textit{texto}\verb!}!               & %
        \textit{Forma it\'alica} \\
\verb!\textsl{!\textit{texto}\verb!}!                    & %
        \verb!{\slshape !\textit{texto}\verb!}!               & %
        \textsl{Forma inclinada} \\
\verb!\textsc{!\textit{texto}\verb!}!                    & %
        \verb!{\scshape !\textit{texto}\verb!}!               & %
        \textsc{Forma caixa alta} \\
\verb!\emph{!\textit{texto}\verb!}!                      & %
        \verb!{\em !\textit{texto}\verb!}!               & %
        \emph{Enfatizado} \\
\verb!\textnormal{!\textit{texto}\verb!}!                & %
        \verb!{\normalfont !\textit{texto}\verb!}!       & %
        \textnormal{Fonte do documento} \\
\verb!\underline{!\textit{texto}\verb!}!                 & %
                                                        & %
        \underline{Sublinhado}
\end{tabular}

O comando da forma (t\textit{tt}t) \verb!(t\textit{tt}t)! lida melhor com espa\c camento do que da forma (t{\itshape tt}t) \verb!(t{\itshape tt}t)!.

\subsection{Font size}
\setlength{\columnsep}{14pt}
\begin{multicols}{2}
\begin{tabbing}
\verb!\footnotesize!          \= \kill
\verb!\tiny!                  \>  \tiny{minusculo} \\
\verb!\scriptsize!            \>  \scriptsize{muito pequena} \\
\verb!\footnotesize!          \>  \footnotesize{nota de rodape} \\
\verb!\small!                 \>  \small{pequena} \\
\verb!\normalsize!            \>  \normalsize{normal} \\
\verb!\large!                 \>  \large{grande} \\
\verb!\Large!                 \=  \Large{maior} \\
\verb!\LARGE!                 \>  \LARGE{muito maior} \\
\verb!\huge!                  \>  \huge{enorme} \\
\verb!\Huge!                  \>  \Huge{gigante}
\end{tabbing}
\end{multicols}
\setlength{\columnsep}{1pt}

Essas declara\c c\~oes devem ser utilizadas da forma 
\verb!{\small! \ldots\verb!}!, ou sem as chaves para aplicar em todo o documento.


\subsection{Texto Verbatim}

\settowidth{\MyLen}{\texttt{.begin.verbatim..} }
\begin{tabular}{@{}p{\the\MyLen}%
                @{}p{\linewidth-\the\MyLen}@{}}
\verb@\begin{verbatim}@ & Ambiente verbatim. \\
\verb@\begin{verbatim*}@ & Espa\c cos s\~ao mostrados com \verb*@ @. \\
\verb@\verb!texto!@ & Texto entre os caracteres delimitadores (neste caso %
                      `\texttt{!}', pode-se usar \texttt{|} tamb\'em).
\end{tabular}


\subsection{Alinhamento}
\begin{tabular}{@{}ll@{}}
\textit{Ambiente}  &  \textit{Declara\c c\~ao}  \\
\verb!\begin{center}!      & \verb!\centering!  \\
\verb!\begin{flushleft}!  & \verb!\raggedright! \\
\verb!\begin{flushright}! & \verb!\raggedleft!  \\
\end{tabular}

\subsection{Miscel\^anea}
\verb!\linespread{!$x$\verb!}! altera o espa\c co entre linhas por um m\'ultiplo de $x$.





\section{S\'imbolos modo texto}

\subsection{S\'imbolos}
\begin{tabular}{@{}l@{\hspace{1em}}l@{\hspace{2em}}l@{\hspace{1em}}l@{\hspace{2em}}l@{\hspace{1em}}l@{\hspace{2em}}l@{\hspace{1em}}l@{}}
\&              &  \verb!\&! &
\_              &  \verb!\_! &
\ldots          &  \verb!\ldots! &
\textbullet     &  \verb!\textbullet! \\
\$              &  \verb!\$! &
\^{}            &  \verb!\^{}! &
\textbar        &  \verb!\textbar! &
\textbackslash  &  \verb!\textbackslash! \\
\%              &  \verb!\%! &
\~{}            &  \verb!\~{}! &
\#              &  \verb!\#! &
\S              &  \verb!\S! \\
\end{tabular}

\subsection{Acentos}
\begin{tabular}{@{}l@{\ }l|l@{\ }l|l@{\ }l|l@{\ }l|l@{\ }l@{}}
\`o   & \verb!\`o! &
\'o   & \verb!\'o! &
\^o   & \verb!\^o! &
\~o   & \verb!\~o! &
\=o   & \verb!\=o! \\
\.o   & \verb!\.o! &
\"o   & \verb!\"o! &
\c o  & \verb!\c o! &
\v o  & \verb!\v o! &
\H o  & \verb!\H o! \\
\c c  & \verb!\c c! &
\d o  & \verb!\d o! &
\b o  & \verb!\b o! &
\t oo & \verb!\t oo! &
\oe   & \verb!\oe! \\
\OE   & \verb!\OE! &
\ae   & \verb!\ae! &
\AE   & \verb!\AE! &
\aa   & \verb!\aa! &
\AA   & \verb!\AA! \\
\o    & \verb!\o! &
\O    & \verb!\O! &
\l    & \verb!\l! &
\L    & \verb!\L! &
\i    & \verb!\i! \\
\j    & \verb!\j! &
!`    & \verb!~`! &
?`    & \verb!?`! &
\end{tabular}


\subsection{Delimitadores}
\begin{tabular}{@{}l@{\ }ll@{\ }ll@{\ }ll@{\ }ll@{\ }ll@{\ }l@{}}
`       & \verb!`!  &
``      & \verb!``! &
\{      & \verb!\{! &
\lbrack & \verb![! &
(       & \verb!(! &
\textless  &  \verb!\textless! \\
'       & \verb!'!  &
''      & \verb!''! &
\}      & \verb!\}! &
\rbrack & \verb!]! &
)       & \verb!)! &
\textgreater  &  \verb!\textgreater! \\
\end{tabular}

\subsection{Tra\c cos}
\begin{tabular}{@{}llll@{}}
\textit{Nome} & \textit{Fonte} & \textit{Exemplo} & \textit{Uso} \\
h\'ifen  & \verb!-!   & Raio-x          & Em textos. \\
en-tra\c co & \verb!--!  & 1--5           & Entre n\'umeros. \\
em-tra\c co & \verb!---! & Sim---ou n\~ao?    & Pontua\c c\~ao.
\end{tabular}


\subsection{Quebra de linha e de p\'agina}
\settowidth{\MyLen}{\texttt{.pagebreak} }
\begin{tabular}{@{}p{\the\MyLen}%
                @{}p{\linewidth-\the\MyLen}@{}}
\verb!\\!          &  Inicia uma nova linha sem novo par\'agrafo.  \\
\verb!\\*!         &  Proibe quebra de p\'agina ap\'os quebra de linha. \\
\verb!\kill!       &  N\~ao imprime linha atual. \\
\verb!\pagebreak!  &  Inicia nova p\'agina. \\
\verb!\noindent!   &  N\~ao indenta linha atual.
\end{tabular}


\subsection{Miscel\^anea}
\settowidth{\MyLen}{\texttt{.rule.w..h.} }
\begin{tabular}{@{}p{\the\MyLen}%
                @{}p{\linewidth-\the\MyLen}@{}}
\verb!\today!  &  \today. \\
\verb!$\sim$!  &  Imprime $\sim$ em vez de \verb!\~{}!, o que torna \~{}. \\
\verb!~!       &  Espa\c co, n\~ao permite quebra de linha (\verb!W.J.~Clinton!). \\
\verb!\@.!     &  Indica que o \verb!.! no final de senten\c ca seguindo uma letra mai\'uscula. \\
\verb!\hspace{!$l$\verb!}! & Espa\c co horizontal de comprimento $l$
                                (Ex: $l=\mathtt{20pt}$). \\
\verb!\vspace{!$l$\verb!}! & Espa\c co vertical de comprimento $l$. \\
\verb!\rule{!$w$\verb!}{!$h$\verb!}! & Linha de largura $w$ e altura $h$. \\
\end{tabular}



\section{Ambientes de tabela}

\subsection{Ambiente \texttt{tabbing}}
\begin{tabular}{@{}l@{\hspace{1.5ex}}l@{\hspace{10ex}}l@{\hspace{1.5ex}}l@{}}
\verb!\=!  &   Define parada de tabula\c c\~ao. &
\verb!\>!  &   Vai para parada de tab.
\end{tabular}

Tabula\c c\~ao pode ser definida com linhas ``invis\'iveis'' com \verb!\kill! no final da linha.  Normalmente \'e usado \verb!\\! para separar linhas.


\subsection{Ambiente \texttt{tabular}}
\verb!\begin{array}[!\textit{pos}\verb!]{!\textit{cols}\verb!}!   \\
\verb!\begin{tabular}[!\textit{pos}\verb!]{!\textit{cols}\verb!}! \\
\verb!\begin{tabular*}{!\textit{largura}\verb!}[!\textit{pos}\verb!]{!\textit{cols}\verb!}!


\subsubsection{\texttt{tabular} especifica\c c\~ao da coluna}
\settowidth{\MyLen}{\texttt{p}\{\textit{width}\} \ }
\begin{tabular}{@{}p{\the\MyLen}@{}p{\linewidth-\the\MyLen}@{}}
\texttt{l}    &   Coluna alinhada a esquerda.  \\
\texttt{c}    &   Coluna centralizada.  \\
\texttt{r}    &   Coluna alinhada a direita. \\
\verb!p{!\textit{width}\verb!}!  &  Mesmo que %
                              \verb!\parbox[t]{!\textit{largura}\verb!}!. \\ 
\verb!@{!\textit{decl}\verb!}!   &  Insira \textit{decl} em vez de 
                                    espa\c co entre colunas. \\
\verb!|!      &   Insere uma linha vertical entre colunas. 
\end{tabular}


\subsubsection{\texttt{tabular} elementos}
\settowidth{\MyLen}{\texttt{.cline.x-y..}}
\begin{tabular}{@{}p{\the\MyLen}@{}p{\linewidth-\the\MyLen}@{}}
\verb!\hline!           &  Linha horizontal entre linhas.  \\
\verb!\cline{!$x$\verb!-!$y$\verb!}!  &
                        Linha horizontal nas colunas de $x$ a $y$. \\
\verb!\multicolumn{!\textit{n}\verb!}{!\textit{cols}\verb!}{!\textit{texto}\verb!}! \\
        & Uma c\'elula que se estende por \textit{n} colunas, com \textit{cols} especifica\c c\~ao de colunas.
\end{tabular}

\section{Modo matem\'atico}
Para matem\'atica na linha, use \verb!$...$!.
Para matem\'atica destacada, use \verb!\[...\]! or \verb!\begin{equation}!.

\begin{tabular}{@{}l@{\hspace{1em}}l@{\hspace{2em}}l@{\hspace{1em}}l@{}}
Superescrito$^{x}$       &
\verb!^{x}!             &  
Subescrito$_{x}$         &
\verb!_{x}!             \\  
$\frac{x}{y}$           &
\verb!\frac{x}{y}!      &  
$\sum_{k=1}^n$          &
\verb!\sum_{k=1}^n!     \\  
$\sqrt[n]{x}$           &
\verb!\sqrt[n]{x}!      &  
$\prod_{k=1}^n$         &
\verb!\prod_{k=1}^n!    \\ 
\end{tabular}

\subsection{S\'imbolos modo matem\'atico}

\begin{tabular}{@{}l@{\hspace{1ex}}l@{\hspace{1em}}l@{\hspace{1ex}}l@{\hspace{1em}}l@{\hspace{1ex}} l@{\hspace{1em}}l@{\hspace{1ex}}l@{}}
$\leq$          &  \verb!\leq!  &
$\geq$          &  \verb!\geq!  &
$\neq$          &  \verb!\neq!  &
$\approx$       &  \verb!\approx!  \\
$\times$        &  \verb!\times!  &
$\div$          &  \verb!\div!  &
$\pm$           & \verb!\pm!  &
$\cdot$         &  \verb!\cdot!  \\
$^{\circ}$      & \verb!^{\circ}! &
$\circ$         &  \verb!\circ!  &
$\prime$        & \verb!\prime!  &
$\cdots$        &  \verb!\cdots!  \\
$\infty$        & \verb!\infty!  &
$\neg$          & \verb!\neg!  &
$\wedge$        & \verb!\wedge!  &
$\vee$          & \verb!\vee!  \\
$\supset$       & \verb!\supset!  &
$\forall$       & \verb!\forall!  &
$\in$           & \verb!\in!  &
$\rightarrow$   &  \verb!\rightarrow! \\
$\subset$       & \verb!\subset!  &
$\exists$       & \verb!\exists!  &
$\notin$        & \verb!\notin!  &
$\Rightarrow$   &  \verb!\Rightarrow! \\
$\cup$          & \verb!\cup!  &
$\cap$          & \verb!\cap!  &
$\mid$          & \verb!\mid!  &
$\Leftrightarrow$   &  \verb!\Leftrightarrow! \\
$\dot a$        & \verb!\dot a!  &
$\hat a$        & \verb!\hat a!  &
$\bar a$        & \verb!\bar a!  &
$\tilde a$      & \verb!\tilde a!  \\

$\alpha$        &  \verb!\alpha!  &
$\beta$         &  \verb!\beta!  &
$\gamma$        &  \verb!\gamma!  &
$\delta$        &  \verb!\delta!  \\
$\epsilon$      &  \verb!\epsilon!  &
$\zeta$         &  \verb!\zeta!  &
$\eta$          &  \verb!\eta!  &
$\varepsilon$   &  \verb!\varepsilon!  \\
$\theta$        &  \verb!\theta!  &
$\iota$         &  \verb!\iota!  &
$\kappa$        &  \verb!\kappa!  &
$\vartheta$     &  \verb!\vartheta!  \\
$\lambda$       &  \verb!\lambda!  &
$\mu$           &  \verb!\mu!  &
$\nu$           &  \verb!\nu!  &
$\xi$           &  \verb!\xi!  \\
$\pi$           &  \verb!\pi!  &
$\rho$          &  \verb!\rho!  &
$\sigma$        &  \verb!\sigma!  &
$\tau$          &  \verb!\tau!  \\
$\upsilon$      &  \verb!\upsilon!  &
$\phi$          &  \verb!\phi!  &
$\chi$          &  \verb!\chi!  &
$\psi$          &  \verb!\psi!  \\
$\omega$        &  \verb!\omega!  &
$\Gamma$        &  \verb!\Gamma!  &
$\Delta$        &  \verb!\Delta!  &
$\Theta$        &  \verb!\Theta!  \\
$\Lambda$       &  \verb!\Lambda!  &
$\Xi$           &  \verb!\Xi!  &
$\Pi$           &  \verb!\Pi!  &
$\Sigma$        &  \verb!\Sigma!  \\
$\Upsilon$      &  \verb!\Upsilon!  &
$\Phi$          &  \verb!\Phi!  &
$\Psi$          &  \verb!\Psi!  &
$\Omega$        &  \verb!\Omega!  
\end{tabular}
\footnotesize

\section{Bibliografia e cita\c c\~oes}
Enquanto usar \BibTeX, voc\^e vai precisar rodar \texttt{latex}, \texttt{bibtex},
e \texttt{latex} mais duas vezes para resolver as depend\^encias.

\subsection{Tipos de cita\c c\~oes}
\settowidth{\MyLen}{\texttt{.shortciteN.key..}}
\begin{tabular}{@{}p{\the\MyLen}@{}p{\linewidth-\the\MyLen}@{}}
\verb!\cite{!\textit{chave}\verb!}!       &
        Lista autor completo e ano. (Watson e Crick 1953) \\
\verb!\citeA{!\textit{chave}\verb!}!      &
        Lista autor completo. (Watson and Crick) \\
\verb!\citeN{!\textit{chave}\verb!}!      &
        Lista autor completo e ano. Watson e Crick (1953) \\
\verb!\shortcite{!\textit{chave}\verb!}!  &
        Lista autor abreviado e ano. ? \\
\verb!\shortciteA{!\textit{chave}\verb!}! &
        Lista autor abreviado. ? \\
\verb!\shortciteN{!\textit{chave}\verb!}! &
        Lista autor abreviado e ano. ? \\
\verb!\citeyear{!\textit{chave}\verb!}!   &
        Cita somente o ano. (1953) \\
\end{tabular}

Todos acima tem uma varia\c c\~ao \texttt{NP} sem par\^enteses;
Ex. \verb!\citeNP!.


\subsection{\BibTeX\ tipos de entrada}
\settowidth{\MyLen}{\texttt{.mastersthesis.}}
\begin{tabular}{@{}p{\the\MyLen}@{}p{\linewidth-\the\MyLen}@{}}
\verb!@article!         &  Artigo de jornal ou revista. \\
\verb!@book!            &  Livro com editora. \\
\verb!@booklet!         &  Livro sem editora. \\
\verb!@conference!      &  Artigo em atas de confer\^encia. \\
\verb!@inbook!          &  Uma parte de um livro e/ou intervalo de p\'aginas. \\
\verb!@incollection!    &  Uma parte de um livro com seu pr\'oprio t\'itulo. \\
\verb!@misc!            &  Se nada mais se encaixar. \\
\verb!@phdthesis!       &  Tese PhD. \\
\verb!@proceedings!     &  Procedimentos de uma confer\^encia. \\
\verb!@techreport!      &  Reportagem t\'ecnica, usualmente numerada em s\'erie. \\
\verb!@unpublished!     &  In\'edito. \\
\end{tabular}

\subsection{\BibTeX\ campos}
\settowidth{\MyLen}{\texttt{organization.}}
\begin{tabular}{@{}p{\the\MyLen}@{}p{\linewidth-\the\MyLen}@{}}
\verb!address!         &  Endere\c co da editora.  \\
\verb!author!           &  Nome dos autores. \\
\verb!booktitle!        &  T\'itulo do livro quando parte dele \'e citado. \\
\verb!chapter!          &  Cap\'itulo ou n\'umero da se\c c\~ao. \\
\verb!edition!          &  Edi\c c\~ao do livro. \\
\verb!editor!           &  Nome da editora. \\
\verb!institution!      &  Institui\c c\~ao patrocinadora do relat\'orio t\'ecnico. \\
\verb!journal!          &  Nome do jornal. \\
\verb!key!              &  Usado para refer\^encia cruzada quando n\~ao h\'a autor. \\
\verb!month!            &  M\^es de publica\c c\~ao. Use abrevia\c c\~ao de 3 letras. \\
\verb!note!             &  Qualquer informa\c c\~ao adicional. \\
\verb!number!           &  N\'umero do jornal ou revista. \\
\verb!organization!     &  Organiza\c c\~ao que patrocina a confer\^encia. \\
\verb!pages!            &  Intervalo de p\'aginas (\verb!2,6,9--12!). \\
\verb!publisher!        &  Nome da editora. \\
\verb!school!           &  Nome da escola (para teses). \\
\verb!series!           &  Nome da s\'erie de livros. \\
\verb!title!            &  T\'itulo do trabalho. \\
\verb!type!             &  Tipo de relat\'orio t\'ecnico, ex. ``Nota de Pesquisa''. \\
\verb!volume!           &  Volume do jornal ou livro. \\
\verb!year!             &  Ano de publica\c c\~ao. \\
\end{tabular}
Nem todos os campos precisam ser preenchidos. Veja o exemplo abaixo.

\subsection{\BibTeX\ arquivo de estilos comuns}
\begin{tabular}{@{}l@{\hspace{1em}}l@{\hspace{3em}}l@{\hspace{1em}}l@{}}
\verb!abbrv!    &  Padr\~ao &
\verb!abstract! &  \texttt{alpha} com resumo \\
\verb!alpha!    &  Padr\~ao &
\verb!apa!      &  APA \\
\verb!plain!    &  Padr\~ao &
\verb!unsrt!    &  N\~ao ordenado \\
\end{tabular}

O documento \LaTeX\ deve ter as duas linhas seguintes antes de \verb!\end{document}!, onde \verb!refs.bib! \'e o nome do arquivo \BibTeX.
\begin{verbatim}
\bibliographystyle{plain}
\bibliography{refs}
\end{verbatim}

\subsection{\BibTeX\ exemplo}
O banco de dados \BibTeX\ fica num arquivo chamado \textit{refs}\texttt{.bib}, que \'e processado com \verb!bibtex refs!. 
\begin{verbatim}
@String{N = {Na\-ture}}
@Article{WC:1953,
  author  = {James Watson and Francis Crick},
  title   = {A structure for Deoxyribose Nucleic Acid},
  journal = N,
  volume  = {171},
  pages   = {737},
  year    = 1953
}
\end{verbatim}

\section{Novo comando}

\verb!\newcommand{\nomecomando}[quant][valor]{comandos}!

onde \texttt{quant} \'e a quantidade de vari\'aveis e \texttt{valor} \'e o valor padr\~ao usado na primeira vari\'avel (opcional).

Exemplo:

\verb!\newcommand{\soma}[2][n]{#2_1 + #2_2 + \ldots + #2_{#1}}!

Uso: \verb!\soma[k]{a}! (no modo matem\'atico).

Um exemplo mais simples:

\verb!\newcommand{\sse}{\Leftrightarrow}!

\newcommand{\sse}{\Leftrightarrow}

Uso: \verb!$\sse$! produz $\sse$


% Com \usepackage[utf8]{inputenc} ou \usepackage[latin1]{inputenc} (Win) pode-se usar acentuação normalmente.

% verbatim não aceita acentuação.
\section{Exemplo de documento \LaTeX}
\begin{verbatim}
\documentclass[a4paper,10pt]{article}
\usepackage[utf8]{inputenc}
\usepackage[brazil]{babel}
\usepackage[cm]{fullpage}
\title{Modelo}
\author{Nome}
\begin{document}
  \maketitle
\section{secao}
  \subsection*{subsecao sem numero}
texto \textbf{negrito} texto. Um pouco de matematica: $2+3=5$
  \subsection{subsecao}
texto \emph{texto enfatizado} texto. \cite{WC:1953}
descoberta a estrutura do DNA.

\begin{table}[!th]
  \centering
  \begin{tabular}{|l|c|r|}
    \hline
    primeira  &  linha  &  valor \\
    segunda &  linha  &  valor \\
    \hline
  \end{tabular}
  \caption{Esta 'e a legenda}
  \label{ex:tabela}
\end{table}
A tabela 'e numerada \ref{ex:tabela}.
\end{document}
\end{verbatim}

\rule{0.3\linewidth}{0.25pt}
\scriptsize

Copyright \copyright\ 2012 Regis Santos

http://latexbr.blogspot.com.br/

original de Winston Chang

http://www.stdout.org/$\sim$winston/latex/


\end{multicols}
\end{document}
